\documentclass[a4paper]{article}
\usepackage[ngerman]{babel}
\usepackage{ucs}
\usepackage[utf8x]{inputenc}
\usepackage[T1]{fontenc}
\usepackage{amsmath,amssymb,amstext}
\usepackage{xcolor}
\usepackage{hyperref}
\usepackage{musixtex}
\usepackage{harmony}

\begin{document}
	\paragraph{Vorwort:}
	Die Informationen der Vorlesung wurden nicht unbedingt in einer strukturierten Form wieder gegeben. Es wird versucht Teils eine Struktur oder einen Überbegriff zu finden und Informationen nicht chronologisch zu Überbegriffen zu zuordnen.\\
	
	\part{Lautsprecher und Lautsprechersysteme}
	Die Kombination aus \textbf{Lautsprecherchassis} und \textbf{Lautsprechergehäuse} bildet die oft irrtümlich als "Lautsprecher" bezeichnete Lautsprecherbox.
	
	\section{Lautsprecherboxen}
	Man kann Lautsprecherboxen nach zwei Kriterien unterscheiden: Wege oder Speisung.
	
	\paragraph{Unterteilung nach Wegen} Werden mehrere Lautsprecherchassis verwendet um unterschiedliche Frequenzbereiche abzudecken, so spricht man von einer $n$-Wege Lautsprecherbox. $n$ bezeichnet hierbei die Anzahl der verschiedenen Frequenzbereiche bzw. normalerweise gleichzeitig die Anzahl der verschiedenen Lautsprecherchassis die in der Lautsprecherbox verbaut sind.\\
	Beispielsweise sind viele der heute üblichen Studiolautsprecherboxen (auch Monitore/Monitorlautsprecherboxen genannt) 2-Wege Lautsprecherboxen, aufgeteilt in eine
	\begin{description}
		\item[Hochtönerlautsprecherbox]("Hochtöner") für die Frequenzen üblicherweise von etwa $3,5$ kHz-$20$ kHz\\ \textit{[Genaue Angaben bei der unteren Grenze nicht garantiert, abgelesen aus einer Grafik im Aufschrieb]} und einer 
		\item[Tieftönerlautsprecherbox]("Tieftöner") für die Frequenzen üblicherweise von $20$ Hz-$5$ kHz.
	\end{description}
	Die Aufteilung des Signals für die einzelnen Lautsprecherchassis erfolgt durch eine \textbf{Frequenzweiche}, die das Signal filtert, sodass an die jeweilige Chassis nur der Frequenzbereich des Signals ankommt, den die Chassis abbilden soll.\\
	Der Bereich von $3,5$ kHz-$5$ kHz heißt \textbf{Übernahmebereich} und liegt innerhalb des \textbf{Präsenzbereichs} von $1,5$ kHz-$8$ kHz, wo das menschliche Ohr besonders empfindlich ist.\\
	Eine \textbf{Ein-Wege Lautsprecherbox} ist bestückt mit einer sogenannten \textbf{Breitbandlautsprecherchassis}.
	
	\section{Unterteilung nach Speisung}
	Mit dem Begriff der Speisung teilt man Lautsprecherboxen in solche ein, die ein vorverstärktes Signal benötigen, da sie keinen eigenen Endstufenverstärker ("{}Endstufe") verbaut haben, und solche die selbst das Signal nochmal verstärken.
	\begin{description}
		\item[Passive Lautsprecherboxen] benötigen ein extern vorverstärktes Signal, da sie selbst nur das Signal (ggf. durch eine Frequenzweiche) an die Lautsprecherchassis weiterleiten. Das Kabel hat meist keinen Stecker sondern nur "lose" (meist verdrehte) Kupferkabelenden, die an den Anschlüssen der Lautprecherbox eingeklemmt werden.
		\item[Aktive Lautsprecherboxen] benötigen eine Stromversorgung, da sie selbst eingebaute Endstufen für Verstärkung des Signals für die einzelnen Chassis besitzen. Angesteuert werden diese meist über Cinch, Klinke oder XLR, aber auch digitale Signale oder sogar drahlose Ansteuerung ist möglich.
	\end{description}
	Neben Lautsprecherboxen, die möglichst das gesamte Frequenzspektrum abdecken sollen, gibt es auch noch Lautsprecherboxen für die nur bestimmte Frequenzbänder abdecken sollen, bekanntestes Beispiel hier für sind \textbf{Subwoofer}.\\
	Subwoofer sind dedizierte Bass-Lautsprecher, die maximal Frequenzen bis $100$Hz abdecken sollen. Eine weitere Eigenschaft von Subwoofern ist, dass sie nicht im Raum geortet werden sollen, und deshalb aufgrund ihrer Abstrahlung meist an der Wand stehen.
	
	\section{Lautsprechersysteme/Anordnungen}
	
\end{document}